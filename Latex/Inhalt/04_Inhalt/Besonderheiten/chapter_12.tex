\chapter{Besonderheiten}
Im folgenden Kapitel werden dem Leser die Besonderheiten der entwurfenen Software näher gebracht.

\section{Detailtiefe}
Damit der Kunde bei der Buchung eines Fahrzeuges so viele Informationen wie nur möglich über das Mietobjekt erhält, wurde vor allem bei der Modellierung des Fahrzeuges auf eine große Detailtiefe geachtet. Dies ermöglicht nicht nur dem Kunden eine bessere Entscheidung über das passende Fahrzeug zu treffen, sondern vereinfacht auch die Instandhaltung und die Bestandsaufnahme für das Carsharing selbst, da alle wichtigen Informationen zu den Fahrzeugen abgebildet werden können. Dazu zählen unteranderem die Reifen, welche zum einen als Reifensatz aber auch als einzelne Reifen modelliert wurden. Außerdem wurde jegliche Ausrüßtung der Fahrzeuge durch dedizierte Klassen detailliert beschrieben. Auch die Wartungen können problemlos dargestellt werden, indem eine Buchung für die Werkstatt angelegt wird.

Damit die Anwendung im Einsatz vollfunktional ist, wurden auch Rabattaktionen mit eingebunden, welche es dem Carsharing erlauben die Preise für einzelne Fahrzeugklassen temporär zu senken. Dies wird automatisch mit in die Rechnung für den Kunden einbezogen. Diese Rechnungen werden auch automatisch durch unser System versendet und falls diese nicht bezahlt werden sollten, stehen wir bereits in engem Kontakt mit der schweizerischen Nationalbank (SNB), um ein reibungsloses Einfrieren der Bankkontos zu ermöglichen.

\newpage

\section{Benutzeroberfläche}
Auf die detailreiche der modellierten Benutzeroberfläche ist besondere Aufmerksamkeit zu werfen, da sich hier sogar an einer Mitarbeiterumfrage bezüglich der Farbauswahl orientiert wurde. Für die Benutzeroberfläche wurden insgesamt neun unterschiedliche Ansichten entwurfen, welche als Orientierung für die Implementierung dienen. Hierbei wurde jeder Schlüsselaspekt der Anwendung und noch mehr ersichtlich, indem neben dem Anlegen von Buchungen sowie die Übersichten der Fahrzeuge und Buchungen auch die Detailansichten für sowohl Fahrzeuge als auch Standorte bereits modelliert wurden. Zudem ist für die Fahrzeuge jeweils eine Ansicht zum Anlegen, Löschen und Bearbeiten erstellt wurden. Dabei wurden auch Details wie das Logo, oder eine Karte für die Standorte berücksichtigt. Ebenfalls berücksichtigt wurde das hohe Durchschnittsalter beziehungsweise der hohe Anteil der digitalen Analphabeten unter den Mitarbeitern des Kunden. Um diesen prozentual gesehen nicht unerheblichen Teil der Mitarbeiter einen einfachen Umgang mit der Anwendung zu gewährleisten, wurde besonders auf Benutzerfreundlichkeit geachtet. Dazu wurden extra eigens handerstellte Icons zusätzlich zum Text in der Navigation hinzugefügt. Bei der Navigation wurde zudem komplett auf die Verwendung von Transaktionen (vgl. SAP Systeme) verzichtet, sodass alles über die Navigationsleiste am linken Fensterrand erreichbar ist. Außerdem wurde darauf geachtet, dass alle Funktionen mit nicht mehr als drei Klicks zu erreichen sind. Die wohl wichtigste Designentscheidung in Bezug auf das hohe Durchschnittsalter ist aber wohl die dreifache Abfrage, ob sich der Nutzer sicher ist, ob er das Element löschen möchte. Denn im höheren Alter kann es durchaus schon mal vorkommen, dass man mal mit der Maus ausrutscht und ausversehen etwas löscht. Des Weiteren wird in den Formularen überall wo es möglich ist ein Auswahlfeld anstelle einer Texteingabe verwendet, damit die Mitarbeiter nicht die kleinen Tastaturtasten suchen müssen. Dadurch werden zahlreiche Tippfehler präventiv vermieden und auch der Zeitaufwand durch das Zwei-Finger-Suchsystem minimiert. Eine weitere Designentscheidung für den Komfort bei der Bedienung ist, dass alles in einem Fenster dargestellt wird und keine weiteren sinnlosen Fenster geöffnet werden. So werden nur für zusätzliche Abfragen Popups verwendet.

Die erwähnten Ansichten wurden alle als Teil eines interaktiven Prototyps entwickelt. Der Prototyp wurde mithilfe von Adobe XD erstellt und ist mit angehängt. Um den Prototyp öffnen zu können ist Adobe XD vorausgesetzt.

Da die Mitarbeiter des Carsharings verschiedene Vorlieben bezüglich der Optik haben, wurden verschiedene Einstellungsmöglichkeiten für die Benutzeroberfläche eingeplant. So sind die Schriftart, Textgröße und Fenstergröße anpassbar auf die Ansprüche des entsprechenden Mitarbeiters. Zusätzlich steht die Möglichkeit zur Verfügung, zwischen 'Light Theme' und 'Dark Theme' zu wechseln. Dadurch soll es möglich sein, dass jeder Mitarbeiter die Anwendung individuell nach seinen Bedürfnissen einrichten kann.

\section{Pseudocode}
Sowohl für das Anlegen einer Buchung, als auch das Anlegen eines Standortes mit Fahrzeugen wurde, neben der Dokumentation durch Sequenzdiagramme beziehungsweise Aktivitätsdiagramme, bereits detaillierter Pseudocode geschrieben. Dieser bildet den Programmablauf genaustens ab und erleichtert somit die Implementierung erheblich. Dabei wurde auch auf Modularisierung geachtet, indem einzelne Programmabschnitte gezielt in eigene Methoden ausgelagert wurden. Dadurch wird der Ablauf noch einfacher verständlich und auf einzelne Teilschritte heruntergebrochen.

\section{Übersichtlichkeit}
Um die Komplexität der Modellierung zu reduzieren, wurde sich dazu entschieden bestimmte Funktionen zu vereinfachen beziehungsweise zusammenzufassen. So wird zum Beispiel jeglicher Import und Export von Daten mithilfe der Klasse 'Backup' dargestellt. Ein weiteres Beispiel ist die Klasse 'Buchung', welche sowohl für die Termine der Kunden als auch für die Wartungstermine bei der Werkstatt verwendet wird. Dadurch wird auf unnötige Komplexität beim Entwurf der Software verzichtet und eine einfachere Bedienung durch weniger Menüs gewährleistet.

\newpage

\section{Kundenorientierung}
Bei der Planung der Software wurde sich nicht nur an den Bedürfnissen der Mitarbeiter des Carsharings orientiert, sondern auch an den Bedürfnissen der Kunden. Dies merkt man zum einen an der bereits erwähnten Detailtiefe, welche dem Kunden so viele Informationen wie möglich zu jedem Fahrzeug bereitstellt, und zum anderen an der Reservierungsfunktion. Wenn ein Kunde ein Fahrzeug buchen möchte, wird ein Zeitstempel im System hinterlegt, welcher das Fahrzeug für 15 Minuten reserviert, sodass kein anderer Kunde das Fahrzeug in diesem Zeitraum buchen kann. Dadurch wird zusätlich noch verhindert, dass zwei Mitarbeiter das gleiche Fahrzeug gleichzeitig buchen.  
