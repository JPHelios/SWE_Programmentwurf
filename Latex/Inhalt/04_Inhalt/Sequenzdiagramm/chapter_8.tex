\chapter{Sequenzdiagramm}

Im Folgenden wird der Vorgang eines vollständigen Buchungsdurchgangs betrachtet, um final mittels Sequenzdiagrammen das Vorgehen visuell darzustellen.

Bei der Erarbeitung wird aufgrund des Umfangs ein Teil der Detailtiefe entfernt. Die exakte Kommunikation mit der Datenbasis (in Realität mit einer Datenbank und im Projekt mit den CSV-Dateien) wird nicht betrachtet. Bei der Interaktion mit der Anwendung wird die Unterteilung in verschiedene GUIs unter dem Begriff 'Benutzeroberfläche' zusammengefasst. Es wird ebenfalls davon ausgegangen, dass für den Sequenzablauf benötigte Datensätze wie Fahrzeuge, Standorte, Kunden und Mitarbeiter bereits vorhanden sind und nicht erst angelegt werden müssen. Davon auszugehen, dass mit einer vollständig leeren Datenbasis begonnen wird ist für die Abbildung des Buchungsablaufs nicht zielführend. Aus diesem Grund wird angenommen, dass ausschließlich die Datenbasis der Buchungen, Rechnungen und Mahnungen leer ist. Weiterhin wird in den Sequenzdiagrammen auf die Verwendung von Funktionen verzichtet und Vorgänge werden umschrieben oder umgangssprachlich aufgeführt.

\newpage

\section{Aktionsbetrachtung: Buchung eines Fahrzeugs}

Eine Buchung soll angelegt werden. Abgesehen vom eigentlichen Buchungsdurchgangs soll auch die Abrechnung des Buchungstermins einbezogen werden. Ausgehend davon lassen sich mehrere Teil-Abläufe identifizieren: Buchung eines Fahrzeugs, Stornierung einer Buchung, Antreten des gebuchten Termins, Beendigung der Fahrt mit Erstellung einer Rechnung und notfalls Mahnungen. Die Buchung und die Stornierung lassen sich zusammenfassen, gleiches gilt für die Beendigung und die Rechnungsausstellung. Da die Buchung über die Desktopanwendung ausschließlich in der Filiale stattfinden kann, ist ein Akteur der Organisator, der zweite Aktuer ist der Kunde.


Für die Buchung und Stornierung interagieren die Akteure mit der Benutzeroberfläche und der Datenbasis. Beim Fahrtantritt findet die Interaktion mit dem Kartenlesegerät und der Datenbasis ab. Das Ende der Fahrt inkludiert neben dem Lesegerät und der Datenbasis einen Server, die Rechnung und Mahnung, sowie den Kunden als Datensatz im System und nicht nur als Akteur.


%Der gesamte Buchungsvorgang beginnt damit, dass ein Kunde die Filiale betritt und einen Termin buchen möchte. Sobald der Kunde einen Terminwunsch und die damit verbundenen Bedinungen (Fahrzeug, Zeitraum, Standort) formuliert, kann der Organisator diese Daten in die Benutzeroberfläche eingeben und die Eingabe validieren lassen. Das System meldet anschließend zurück, ob die Eingabekombination buchbar ist oder nicht.


%Die Stornierung eines gebuchten Termins ist bis zu 10 Stunden vor Antritt möglich. Um einen Termin stornieren zu können, muss der Kunde telefonisch oder in Person die Stornierung beim Organisator anfragen. Dieser filtert nach dem Kunden und wählt die betroffene Buchung aus. Sobald die richtige Buchung gefunden wurde, kann sie gelöscht werden.


%Beim Terminantritt muss der Kunde seine Kundekarte dem Kartenlesegerät präsentieren, welches zum gebuchten Fahrzeug gehört. Der darin enthaltene Mini-Controller vergleicht die in der Karte enthaltenen Kundendaten mit den Daten in der Datenbasis. Sofern für den Kunden eine Buchung hinterlegt ist, bei der das Fahrzeug und die Zeit übereinstimmen, wird das Fahrzeug vom Controller entsperrt.


%Nach der finalen Abgabe überprüft der Controller den Kilometerstand und verriegelt anschließend das Fahrzeug. Serverseitig wird nun der neue Kilometerstand mit dem alten Stand verglichen und daraus wird die gefahrene Kilometeranzahl berechnet. Auf Basis der Buchung (enthält gebuchtes Fahrzeug, welches wieder auf die Preiskalssen verweist) und den Kilometern wird ein Rechnungsobjekt erstellt. Letztendlich erhält der Kunde per E-Mail einen visuellen Export des Rechnungsobjekts.


%Der Kunde hat nun standardmäßig 30 Tage Zeit, um die Rechnung zu zahlen. Sollte die Frist versäumt werden, wird die erste Mahnung erstellt und dem Kunden zugesandt. Nach 45 Tagen erfolgt die zweite Mahnung. Sollte nach zwei Monaten die Rechnung immer noch ausstehend sein, wird das betroffene Kundenkonto gesperrt und rechtliche Schritte werden eingeleitet.

\newpage

\newpage

\section{Pseudo-Code}
Im folgenden Abschnitt wird der Pseudocode für das Anlegen einer Buchung anhand der Methode BUCHUNG-ANLEGEN beschrieben. Da das Anlegen einer Buchung einen großen Umfang hat werden bestimmte Abschnitte in separate Methoden ausgelagert, welche ebenfalls erläutert werden.

\lstinputlisting[firstline=3, lastline= 47,style=Pseudocode, caption={Szenario Buchung anlegen}]{Quellcode/Pseudocode.txt}
Wenn der Kunde einen Termin wunsch bei einem Organisator äußert startet die Methode BUCHUNG-ANLEGEN. Zuerst öffnet der Organisator das Buchungsformular. Daraufhin wird dem Organisator angezeigt, dass er alle Eingabefelder ausfüllen muss, um eine Buchung anzulegen. Durch Interaktion mit dem Kunden füllt der Organisator nun das Formular nach den Wünschen des Kunden aus. Dazu wird der Kunde, ein Standort sowie Fahrzeug, ein Start- und Endtermin ausgewählt. Optional ist auch das auswählen einer Rabattaktion möglich. Falls der Kunde keinen Rabattcode hat, wählt der Organisator als Rabattaktion '-' aus. Wenn der Kunde noch nicht registriert ist muss zuerst die Methode KUNDE-ANLEGEN ausgeführt werden. Diese Methode wird im Anschluss genauer erklärt. Um Fehler bei der Eingabe zu vermeiden, wird stets überprüft, ob der Starttermin vor dem Endtermin liegt und bei der Auswahl des Fahrzeuges muss der Organisator zuerst den Standort angeben. Nachdem der Standort angegeben wurde, werden alle Fahrzeuge des entsprechenden Standortes geladen. Somit wird verhindert, dass der Organisator ein Fahrzeug von einem verkehrtem Standort auswählt.
Sobald alle Felder ausgefüllt sind kann der Organisator auf 'Buchung überprüfen' klicken. Daraufhin wird überprüft, ob das Fahrzeug für den angegebenen Zeitraum verfügbar ist. Sollte das Fahrzeug nicht verfügbar sein, so wählt der Organisator ein neues Fahrzeug aus bis der Zeitraum passt. Wenn das Fahrzeug verfügbar ist, wird die Buchung gespeichert. Nun hat der Kunde noch die Option die Buchung wieder zu stornieren. Sollte sich der Kunde rechtzeitig dazu entscheiden die Buchung zu stornieren, so wird die Methode 'BUCHUNG-STORNIEREN' ausgeführt. Falls der Kunde die Buchung nicht stornieren möchte terminiert die Methode.

\lstinputlisting[firstline=50, lastline= 69,style=Pseudocode, caption={Pseudocode für das Anlegen eines Kunden}]{Quellcode/Pseudocode.txt} \label{code:KundeAnlegen}
Diese Methode wird ausgeführt, wenn der Kunde noch nicht im System gespeichert ist. Zuerst öffnet sich hierbei das Formular für das Anlegen eines neuen Kunden. Solange der Organisator nicht alle Felder ausgefüllt hat, wird eine Fehlermeldung angezeigt, dass alle Eingabefelder ausgefüllt sein müssen um einen Kunden anzulegen. In Zusammenarbeit mit dem Kunden trägt der Organisator nun den Vornamen, Nachnamen, die Adresse, die Mail-Adresse, die Telefonnummer und die Nummer des Schweizer Bankkontos des Kunden ein. Wenn alle Felder ausgefüllt wurden, kann der Organisator auf 'Kunde anlegen' klicken. Daraufhin wird das Schweizer Bankkonto auf validität überprüft, bevor eine Mail an die angegebene Mail-Adresse gesendet wird. Dadurch wird eine Zwei-Faktor-Authentifizierung ermöglicht. Der Kunde klickt nun auf 'Verifizieren' in der Mail, woraufhin die Daten gespeichert werden. Falls das Schweizer Bankkonto nicht valide ist, wird automatisch eine Mail an die Polizei geschickt, um Anzeige wegen Betrugs zu erstatten. 

\lstinputlisting[firstline=72, lastline= 83,style=Pseudocode, caption={Pseudocode für das Stornieren einer Buchung}]{Quellcode/Pseudocode.txt}
Die Methode BUCHUNG-STORNIEREN wird immer dann ausgeführt, wenn eine Buchung bereits angelegt wurd und der Kunde sich dazu entscheidet, diese Buchung rückgängig zu machen. Hierzu klickt der Organisator zuerst auf 'Kunden'. Dadurch wird die Übersicht aller Kunden geöffnet. Nun kann der Organisator den Kunden auswählen, der seine Buchung stornieren möchte. Sobald der Organisator einen Kunden ausgewählt hat wird die Detailansicht für diesen Kunden geöffnet. Hier sind auch die Buchungen des Kunden zu sehen. Der Organisator wählt eine Buchung aus. Daraufhin werden die Details der Buchung angezeigt. Hier kann der Organisator auf 'Stornieren' klicken. Wenn der Organisator auf 'Stornieren' klickt wird ihm eine Warnmeldung angezeigt, bei welcher er bestätigen muss, dass er die Buchung stornieren möchte. Klickt der Organisator erneut auf stornieren, so wird der Datensatz der Buchung als storniert markiert.


\newpage

\section{Diagramme}

\subsection{Buchung anlegen}


Das Anlegen einer Buchung wird in Abbildung \ref{img:buchung01} auf Seite \pageref{img:buchung01} dargestellt. Die Sequenz beginnt damit, dass der Kunde einen Terminwunsch gegenüber einem Mitarbeiter äußert. Der Organisator öffnet daraufhin das Formular für das Anlegen einer Buchung und teilt dem Kunden mit, dass die Buchung nun entgegengenommen werden kann.

Für die eigentliche Buchung nennt der Kunde dem Organisator zuerst den Zeitraum, den Standort und das Fahrzeug. Der Organisator trägt diese Daten in das Formular ein und klickt anschließend auf 'Buchung überprüfen'. Daraufhin lädt die Anwendung die Fahrzeug-, Standort- sowie Buchungsdaten und gleicht diese mit der Eingabe des Organisators ab. Falls es eine Kollision mit einer anderen Buchung gibt, wird die Buchung abgebrochen. In diesem Fall nennt der Kunde andere Buchungsdaten, bis es keine Kollision mehr gibt. Sollte es keine Kollision geben, wird die Buchung in der Datenbasis hinterlegt und das System bestätigt die Buchung. Daraufhin teilt der Organisator dem Kunden mit, dass die Buchung erfolgreich war.

Optional kann der Kunde die Buchung nun stornieren. Dazu muss der Kunde gegenüber dem Organisator den Wunsch äußern, die Buchung zu stornieren. Der Organisator erfagt daraufhin den Namen des Kunden und gibt diesen in der entsprechenden Suchleiste ein. Die Anwendung lädt daraufhin die Kundendaten aus der Datenbasis und zeigt sie dem Organisator an. Im Anschluss erfagt der Organisator, welche Buchung der Kunde gern stornieren möchte. Nachdem der Kunde die Buchung spezifiziert hat, wählt der Organisator die entsprechende Buchung aus. Daraufhin lädt die Anwendung die Buchungsdaten für die spezifizierte Buchung aus der Datenbasis und zeigt diese Daten an. Der Organisator klickt anschließend auf 'Buchung stornieren', wodurch die Buchung aus der Datenbasis gelöscht wird. Der Organisator erhält darufhin eine Bestätigung für das Löschen der Buchung und teilt dem Kunden mit, dass die Buchung erfolgreich storniert wurde. 

\newpage

\begin{figure}[!ht]
    \centering
    \includegraphics[width=\textwidth, height=\textheight-4cm]{Bilder/Diagramme/SD_Buchungsvorgang_01.pdf}
    \caption{Buchung eines Termins}
    \label{img:buchung01}
\end{figure}


\clearpage

\subsection{Fahrt antreten}

Das Antreten einer Fahrt wird in Abbildung \ref{img:buchung02} auf Seite \pageref{img:buchung02} dargestellt. Hierbei lässt der Kunde zuerst seine Kundenkarte vom Lesegerät einscannen, welches die Kundendaten ausliest und im Anschluss die Buchungsdaten aus der Datenbasis lädt. Nachdem die Buchungsdaten geladen wurden, wird zunächst der Standort überprüft, bevor die Kundendaten mit den Buchungsdaten abgeglichen werden. Bei einer übereinstimmung wird zuerst der Kilometerstand für die Abrechnung gespeichert und danach wird das Fahrzeug entriegelt. Nun kann der Kunde seine Fahrt beginnen. 


\begin{figure}[!ht]
    \centering
    \includegraphics[width=\textwidth, trim = 0cm 13cm 0cm 0cm]{Bilder/Diagramme/SD_Buchungsvorgang_02.pdf}
    \caption{Antritt eines gebuchten Termins}
    \label{img:buchung02}
\end{figure}

\newpage

\subsection{Fahrt beenden}

Das Beenden einer Fahrt wird in Abbildung \ref{img:buchung03} auf Seite \pageref{img:buchung03} dargestellt. Sobald der Kunde das Fahrzeug abstellt wird der Kilometerstand vom Lesegerät abgefragt und das Fahrzeug wird verriegelt. Während der Kunde auf seine Rechnung wartet wird der neue Kilometerstand an den Server gesendet, welcher nach Abfrage des alten Kilometerstands und den Buchungsdaten die gefahrenen Kilometer berechnet. Basierend auf den Buchungsdaten und den gefahrenen Kilometern wird eine Rechnung erstellt, welche vom Server per Mail an den Kunden geschickt wird.


\begin{figure}[!ht]
    \centering
    \includegraphics[width=\textwidth, trim = 0cm 14cm 0cm 0cm]{Bilder/Diagramme/SD_Buchungsvorgang_03.pdf}
    \caption{Abschluss eines gebuchten Termins}
    \label{img:buchung03}
\end{figure}

Falls die Rechnung vom Kunden ignoriert wird, erhält der Kunde nach 30 und nach 45 Tagen eine Mahnung. Dies wird in Abbildung \ref{img:buchung03} auf Seite \pageref{img:buchung03} dargestellt. Um eine Mahnung zu versenden, lädt der Server zuerst die Rechnungsdaten der überfälligen Rechnung von der Datenbasis.
Im Anschluss erstellt der Server basierend auf der Rechnung eine Mahnung und sendet diese per Mail an den Kunden. Sollte der Kunde nach 60 Tagen nicht gezahlt haben, so wird er für vom Carsharing-Angebot ausgeschlossen. Dazu lädt der Server zuerst die Mahnungsdaten aus der Datenbasis, um den Status der Mahnung zu überprüfen. Wenn die Rechnung zu diesem Zeitpunkt nicht beglichen wurde, werden die Kundendaten aus der Datenbasis geladen und aktualisiert, sodass der Kunde für zukünftige Buchungen gesperrt ist. Sobald der Kunde blockiert ist, sendet der Server eine Mail an den Kunden mit der Information, dass der Kunde in Zukunft vom Carsharing-Angebot ausgeschlossen ist und sein Schweizer Bankkonto eingefroren wurde.

\begin{figure}[!ht]
    \centering
    \includegraphics[width=\textwidth, trim = 0cm 16cm 0cm 0cm]{Bilder/Diagramme/SD_Buchungsvorgang_04.pdf}
    \caption{Erstellung der Mahnungen}
    \label{img:buchung04}
\end{figure}
