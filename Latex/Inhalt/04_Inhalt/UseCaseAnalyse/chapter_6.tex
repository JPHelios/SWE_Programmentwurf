\chapter{Use-Case Diagramm}

\section{Vorüberlegung}

\begin{enumerate}[itemsep= 0 cm]
    \item Buchung verwalten
    \begin{enumerate}[itemsep= -0.25 cm]
        \item Buchung eines Termins
        \item gebuchten Termin bearbeiten
        \item Stornierung eines Termins
        \item Buchung einsehen
    \end{enumerate}
    \item Kunde verwalten
    \begin{enumerate}[itemsep= -0.25 cm]
        \item neuen Kunden Anlegen
        \item bestehenden Kunden löschen
        \item bestehenden Kunden bearbeiten
        \item bestehenden Kunden einsehen
    \end{enumerate}
    \item Mitarbeiter verwalten
    \begin{enumerate}[itemsep= -0.25 cm]
        \item neuen Mitarbeiter anlegen
        \item bestehenden Mitarbeiter löschen
        \item bestehenden Mitarbeiter einsehen
        \item Mitarbeiterrolle bearbeiten
    \end{enumerate}
    \item Fahrzeug verwalten
    \begin{enumerate}[itemsep= -0.25 cm]
        \item neues Fahrzeug anlegen 
        \item bestehendes Fahrzeug bearbeiten
        \item bestehendes Fahrzeug löschen
        \item bestehendes Fahrzeug einsehen
        \item Wartungstermin festlegen
        \item Fahrzeugbild hochzuladen
        \item Standortveränderung eines Fahrzeugs einplanen
    \end{enumerate}
    \item Rabattaktion verwalten
    \begin{enumerate}[itemsep= -0.25 cm]
        \item neue Rabattaktion anlegen
        \item bestehende Rabattaktion löschen
        \item bestehende Rabattaktion einsehen
    \end{enumerate}
    \item Auswahlliste bearbeiten
    \item Back-Up verwalten
    \begin{enumerate}[itemsep= -0.25 cm]
        \item neues Back-Up erstellen
        \item Daten importieren
        \item Daten exportieren
    \end{enumerate}
    \item Standort verwalten
    \begin{enumerate}[itemsep= -0.25 cm]
        \item neuen Standort anlegen
        \item bestehenden Standort bearbeiten
        \item bestehenden Standort löschen
        \item bestehenden Standort einsehen
    \end{enumerate}
    \item Ausrüstung verwalten
    \begin{enumerate}[itemsep= -0.25 cm]
        \item neuen Ausrüstungsgegenstand anlegen
        \item bestehenden Ausrüstungsgegenstand bearbeiten
        \item bestehenden Ausrüstungsgegenstand löschen
        \item bestehenden Ausrüstungsgegenstand einsehen
    \end{enumerate}
    \item Konfiguration der Benutzeroberfläche ändern
    \item Rechnung verwalten
    \begin{enumerate}[itemsep= -0.25 cm]
        \item neue Rechnung anlegen
        \item bestehende Rechnung archivieren
        \item bestehende Rechnung einsehen
    \end{enumerate}
    \item Mahnung versenden
    \item In System einloggen
\end{enumerate}

\section{Use-Case Analyse}

\textbf{Akteure:}


\begin{itemize}
    \item \textbf{Nutzer}: Der Nutzer ist ein Akteur, der im Rahmen von konzeptionellen Vorüberlegungen entstanden ist. An sich gibt es keinen 'Nutzer' im System, doch beschreibt dieser Akteur die Use-Cases, die von 'Personalmitarbeiter', 'Organisator' und 'Administrator' durchgeführt werden.
    \item \textbf{Personalmitarbeiter}: Ein Personalmitarbeiter ist ein Angestellter des Car-Sharing-Unternehmens, welcher sich um die HR-Verwaltung kümmert. Hauptsächlich arbeitet er mit dem zusätzlichen Verwaltungssystem, doch hat er in der Desktopanwendung die Aufgabe der Verwaltung aller Mitarbeiter und Kundendaten.
    \item \textbf{Organisator}: Organisatoren sind die Mitarbeiter des Unterhemens, die aktiv in der Filiale arbeiten und Kundenkontakt pflegen. Organisatoren bearbeiten alle Anfragen von Kunden in der Filiale und ermöglichen einen ununterbrochenen Geschäftsablauf. 
    \item \textbf{Administrator}: Administratoren sind hauptsächlich für die technischen Aspekte der Firmenarbeit eingestellt und müssen das Funktionieren der Systeme gewährleisten. Sollten Probleme und Fehler auftreten, so ist es die Aufgabe der Administratoren diese Fehler zu beheben. Die Datensicherung fällt auch in diesen Bereich. Weiterhin sollen Administratoren neben ihren eigentlichen Aufgaben trotzdem Zugriff auf alle anderen Daten haben und beispielsweise auch die Rollenvergabe an die anderen Mitarbeiter betreiben.
    \item \textbf{Kunde}: Den Kunden betrifft im Konzept der Desktopanwendung keinen systembezogenen Bedeutung. In einem späteren Stadium des Systems mit einer parallel laufenden Webanwendung soll der Kunde seine eigenen Termine buchen können, doch werden in der Filiale alle Dateneingaben durch die Organisatoren getätigt. Trotzdem ist ein gewisser Datenfluss vom Kunden zu bestimmten Use-Cases wie die Terminbuchung oder die Anmeldung als neuer Kunde nötig (z.B. die Wahl des Terminwunsches oder des Fahrzeugs und das Unterschreiben der Verträge). 
\end{itemize}

\newpage

\textbf{Use-Cases:}


Die Car-Sharing-Anwendung, welche im Rahmen des Programmentwurfs erstellt werden soll ist ausschließlich eine Verwaltungssoftware. Ein Großteil aller möglichen Interaktionsmöglichkeiten mit dem Programm ist das Erstellen und Bearbeiten von verschiedensten Datensätzen. 
In der unten stehenden Tabelle erfolgt zu jedem erstellten Use-Case eine wörtliche Erklärung. Wichtig zu betrachten dabei ist, dass alle aufgeführten Verwaltungsaufgaben mit Ausnahme einiger Fälle auf dem gleichen Konzept basieren: neuen Datensatz erstellen, bestehenden Datensatz bearbeiten, bestehenden Datensatz löschen oder bestehenden Datensatz einsehen.

Die analytische Darstellung der Use-Cases in Diagrammform erfolgt später in zwei Schritten. Der erste Ansatz ist ein  Überblick über die grob definierten Use-Cases. Die erwähnten Use-Cases, die 'verwalten' enthalten sind eine Zusammenfassung aus den Unterfunktionen 'anlegen, bearbeiten, einsehen und löschen'. In einigen Fällen kommen weitere Unter-Use-Cases hinzu. Exemplarisch dafür wird ein zweites Diagramm erstellt, welches diese Aufteilung genauer beschreiben wird. Die Erstellung von allen auftretenden Hierarchiestufen der Use-Case-Diagramme ist im Rahmen des Programmentwurfs nicht zielführend, müsste in einem realen Projekt aber vollständig ausgebaut werden. 

\newpage

\begin{tabular}{l | p{13cm}}
    \hline
    Use-Case 1: & Ein Kunde kann vor Ort in der Filiale einen Termin buchen. Dazu wird der Terminwunsch des Kunden gebraucht und der Filialmitarbeiter trägt den Termin als Buchung ins System ein. Es ist zusätzlich nötig, bestehende Buchungen bearbeiten zu können, sollte sich z.B. der Zeitraum der Buchung auf Wunsch des Kunden ändern. Es existiert auch der Fall, dass eine Buchung storniert wird, was bis zu 10h vor dem gebuchten Termin möglich ist. Die Mitarbeiter sollen natürlich auch bestehende Buchungen einsehen können\\
    \hline
    Use-Case 2 & In Anlehnung an die Erkenntniss, dass alle Verwaltungsaufgaben gleiche Grundaufgaben abdecken müssen, trifft dies auch auf die Verwaltung der Kundendatensätze zu. Ein Unternehmen hat immer einen dynamischen Kundenstamm. Es kommen neue Kunden hinzu, bestehende Kunden benötigen eine Datenanpassung (z.B. bei Änderung von Kontakt- oder Addressinformationen), die Daten müssen für die Mitarbeiter einsehbar sein und auf Kundenwunsch kann der Kunden-Account auch geschlossen werden.\\
    \hline
    Use-Case 3 & Neben den Kunden muss es auch einen Datenstamm mit Mitarbeiterdaten geben. Im Lastenheft wurde hervorgehoben, dass die Datenverwaltung der Mitarbeiter (Gehalt, Zahlungsinformationen, etc.) in einem gesonderten System stattfindet. Bei der Mitarbeiterverwaltung in der Desktopanwendung ist ausschließlich die Verwaltung auf Rollenverteilung vorgesehen. Trotzdem werden dafür die Mitarbeiter auch in diesem System als Datensatz benötigt, aber in einem viel geringeren Umfang als im anderen System.\\
    \hline
\end{tabular}

\begin{tabular}{l | p{13cm}}
    \hline
    Use-Case 4 & Die 'Vermietung' von Fahrzeugen ist der Kernbestandteil des Business-Konzepts des Car-Sharing-Unternehmens. Aus diesem Grund ist der Umfang der Verwaltung von Fahrzeugen umfangreicher als bei anderen Verwaltungsobjekten. Neben den mittlerweile als Standard-Aufgaben klassifizierten Aufgaben benötigen Fahrzeuge weiterhin Wartungstermine (beispielsweise für Reparaturen oder Reifenwechsel). Die Fahrzeuge sind an Standorten abgestellt. Sollten sich zu viele Fahrzeuge an einem Standort befinden, so müssen die Fahrzeuge wieder auf andere Standorte verteilt werden. Schließlich ist noch vorgesehen, dass jeder Fahrzeug-Datensatz Bilder enthält, die das Fahrzeug zeigen. Auch diese Bilder müssen im System hochgeladen werden. \\
    \hline
    Use-Case 5 & Aus verschiedensten Anlässen sollen Rabattaktionen möglich sein. Diese Rabattaktionen müssen grafisch dargestellt werden, sowie müssen alle damit verbundenen Konditionen einsehbar sein. Sollte eine Rabattaktion auslaufen, müssen die damit verbundenen Informationen auch wieder aus dem System entfernt werden.\\
    \hline
    Use-Case 6 & Die Benutzeroberfläche der Anwendung soll nach dem Lastenheft so viele Auswahllisten wie möglich und sinnvoll aufweisen, um Eingabefehler besser vermeiden zu können. Diese Listen sollen ebenfalls einfach editierbar sein, um die Auswahlmöglichkeiten schnell anpassen zu können.\\
    \hline
    Use-Case 7 & Der geforderte Datenimport und -export wurde unter dem Use-Case 'Back-Up verwalten' zusammengefasst. Exportierte Daten sollen entweder direkt als Back-Up verwendet oder für Analyseanwendungen weiterverwendet werden. Der Datenimport bezieht sich hauptsächlich auf das Einspielen von Back-Up Datensätzen.\\
    \hline
\end{tabular}

\begin{tabular}{l | p{13cm}}
    \hline
    Use-Case 8 & Für die Verwaltung von Standorten erfolgen die vier Grundverwaltungsaufgaben. Auch wenn die Standorte weniger dynamisch sind als andere Datensätze, so ist es doch möglich, dass das Unternehmen neue Fahrzeugstandorte aufbaut oder alte Standorte bei zu wenig Auslastung schließt. \\
    \hline
    Use-Case 9 & Die zu verwaltende Ausrüstung ist fahrzeugbezogen. Da das Car-Sharing Unternehmen einen großen Wert auf Kundenzufriedenheit legt und dafür auch mehr anbieten möchte als andere Car-Sharing Unternehmen gibt es zusätzliche Ausrüstung (z.B. Fahrradträger, Dachbox oder Hundetransportbox) die bei verschiedensten Autos an- und abmontiert werden kann. \\
    \hline
    Use-Case 10 & Eine Anforderung an die Desktopanwendung ist die Konfiguration der Benutzeroberfläche. Schriftgröße, Schriftart und Farbmodus sollen wählbar sein. Die Konfiguration findet anwendungsweit (also auf einzelnen Rechnern mit dieser Anwendung) statt.\\
    \hline
    Use-Case 11 & Nach jeder abgeschlossenen Fahrt eines Kunden soll automatisch eine Rechnung erstellt werden. Da die Rechnung ein rechtsgültiges Dokument sein muss, ist die Bearbeitung des generierten Dokuments nicht möglich und auch das Löschen der Rechnung soll nicht möglich sein. Nach Abzahlung der offenen Rechnung soll nur die Archivierung erfolgen, welche über Jahre aufgehoben werden muss.\\ 
    \hline
    Use-Case 12 & Sollte die Zahlung einer ausstehenden Rechnung überfällig sein, soll eine Mahnung zur Zahlung erstellt werden, welche an den Kunden geschickt wird. Die exakten finanziellen Abläufe müssen hier wegen des eigenen Finanzbuchhaltungssystems nicht von belang. \\
    \hline
    Use-Case 13 & Voraussetzung für alle Verwaltungsaufgaben ist eine erfolgreiche Anmeldung am System. Je nach zugewiesener Rolle und der damit verbundenen Rechte dürfen verschiedene Datensätze verwaltet werden oder nicht.\\
    \hline

\end{tabular}

\newpage

\section{Use-Case Diagramm}

\includesvg{Bilder/Diagramme/Use-Case Diagramm.svg}