\chapter{Analyse}

Anmerkung: Einige Fragen wurden bewusst "schwammig" beantwortet. Dahingehend müsste sich in einem wirklichen Gespräch mit dem Auftraggeber ein Dialog ergeben, in dem man schrittweise sich der genauen Beantwortung der Frage nähert. 

\textcolor{YellowGreen}{Text in grün steht für die Fragen der Analyse.}

\textcolor{NavyBlue}{Text in blau steht für die Antworten der Analyse.}

\section{Einleitung}

Für unsere Carsharing-Organisation \emph{Citycar*BaDö} benötigen wir ein neues Buchungssystem, um dem wachsenden Bedarf an gemeinsam genutzten Fahrzeugen gerecht werden zu können. 

\emph{Citycar*BaDö} hat inzwischen fast 500 Mitglieder, denen eine Fahrzeugflotte von ca. 60 Fahrzeugen an ungefähr 30 Standorten in und um Bad Dödelhausen zur Verfügung steht. 

Bisher werden die Mitgliedschaften und Vermietungen mit einem inzwischen in die Jahre gekommenen Buchungssystem verwaltet, das sich recht umständlich bedienen lässt. 

\textcolor{YellowGreen}{Existiert ein Buchhaltungssystem?}

\textcolor{NavyBlue}{Ja, wir verwenden bereits ein extra Buchhaltungssystem für die Verwaltung aller finanztechnischen Aspekte}

\textcolor{YellowGreen}{Welches Buchungssystem haben Sie verwendet?}

\textcolor{NavyBlue}{Wir verwenden ein in die Jahre gekommenes SAP System von 1993.}

\textcolor{YellowGreen}{Was war daran so umständlich?}

\textcolor{NavyBlue}{Neben einer sehr unübersichtlichen und optisch nicht allzu ansprechenden Oberfläche ist die Navigation innerhalb der Anwendung nur mit Transaktionscodes möglich. Sobald man sich in das System eingearbeitet hat, ist es einfacher, doch unsere Mitarbeiter müssen doch regelmäßig bestimmte Codes nachschlagen, wenn Aufgaben anstehen, die weniger zum Alltagsgeschäft gehören. Wir hätten dahingehend gerne einen übersichtliche und auf einen Blick einsehbare Navigation.}

\textcolor{YellowGreen}{Was macht ein Mitglied aus, welche Eigenschaften hat es?}

\textcolor{NavyBlue}{Ein Mitglied ist in dem Sinne nichts weiter als ein Kunde bei anderen Unternehmen, der die Dienstleistung unserer Firma in Anspruch nimmt. In diesem Sinne beschreibt es eine reale Person, mit allen Informationen, die zur Abbildung nötig sind. } %Attribute von Mitgliedern?

\textcolor{YellowGreen}{Welche Anforderungen müssen erfüllt werden, damit man als Mitglied (ins System) aufgenommen werden kann?}

\textcolor{NavyBlue}{Unsere Mitglieder sind natürlich von der SCHUFA als liquide eingestuft und müssen zur Anmeldung eine Bescheinigung vorlegen. Das Mindestalter ist 21 und man muss aus der Probezeit heraus sein. Für die Abwicklung der Transationen wird weiterhin ein schweizer Bankkonto vorausgesetzt.}

Zwar können Buchungen online erfolgen, aber wir würden gerne zusätzliche Informationen für die Online-Kunden zur Verfügung stellen und eine Erweiterung der alten Software lohnt sich nicht.

\textcolor{YellowGreen}{Welches Budget steht für das Projekt zur Verfügung?}

\textcolor{NavyBlue}{In anbetracht des Projektumfangs gedenken wir ein Budget von 8.697.370 Mauritius-Rupien (umgerechnet 187 000 €) bereitzustellen.}

\textcolor{YellowGreen}{Was für zusätzliche Funktionen und Informationen konnte das alte Programm speziell nicht erfüllen?}

\textcolor{NavyBlue}{Neben den Navigationsproblemen war der Datenexport nur in Form von Druckaufträgen bei uns möglich. Das würden wir gerne in Anlehnung an die Datensicherung modifizieren und auch andere Verfahren verwenden, die den Export vereinfachen. Die Informationsdarstellung erfolgte auch nur, indem man sich die betreffenden Texte als .txt-Dateien auf den lokalen Rechner herunterladen konnte. Wir wollen die Informationen in den Anwendung einsehen können.}

\section{Lastenheft}

\subsection{Zielsetzung}

Ziel des Entwicklungsauftrags ist eine Software für die Verwaltung aller Daten, die für die Verwaltung der Fahrzeuge, Kunden sowie Buchungen unserer Carsharing-Organisation anfallen und benötigt werden. 

Alle Daten sollen zentral gespeichert werden, da mehrere Benutzer gleichzeitig auf die Daten und Termine zugreifen werden. 

\textcolor{YellowGreen}{Welche Hardware liegt für diese zentrale Speicherung vor (Server, Speicher, Betriebssystem)?}

\textcolor{NavyBlue}{Im Zusammenhang mit unserem alten System haben wir einen eigenen Server in unserer Firmenzentrale stehen. Laut Angaben unserer IT-Abteilung besitzt der Server 32 GB RAM und 10 TB Speicher. Das laufende Betriebssystem ist Windows XP.
Sollten im Rahmen der Entwicklung Hardware-Anschaffungen nötig sein, haben wir in unserem Budget einen Puffer eingeplant. Es wäre jedoch gut, wenn wir die bereits vorhandene Hardware auch integrieren könnten.}

\textcolor{YellowGreen}{In welcher Form, in welchem Format sollen die Daten zentral gespeichert werden?}

\textcolor{NavyBlue}{Dahingehend haben wir keine speziellen Anforderungen. Wir würden die Methode bevorzugen, die effizient und zugleich auch kostengünstig ist. Eine Datenbank währe denkbar.}

\textcolor{YellowGreen}{Wie viele Benutzer verwenden die Daten gleichzeitig?}

\textcolor{NavyBlue}{Das kann man so pauschal leider nicht sagen. Die Zugriffszahlen hängen von der Tageszeit, Jahreszeit und dem aktuellen Geschehen ab. Auch wenn wir wachsende Kundenzahlen verbuchen ist es bis jetzt jedoch eher unüblich, dass z.B. mehrere Kunden gleichzeitig bei uns in den Filialen sind. }

\textcolor{YellowGreen}{Wie soll mit Kollisionen, gleichzeitigen und gegensätzlichen Zugriffen auf Daten und Terminen umgegangen werden?}

\textcolor{NavyBlue}{Sollte es nicht möglich sein solche Situationen zu umgehen? Ansonsten sollen diese Probleme halt vom System gelöst werden, damit unsere Mitarbeiter normal mit der Anwendung arbeiten können.}

\textcolor{YellowGreen}{Welche Form soll die Software haben? Soll es zum Beispiel eine Webapp oder eine Desktopanwendung sein?}

\textcolor{NavyBlue}{siehe 'Zusammenspiel mit anderen Systemen'}

Ein selektiver Import und Export von Daten über lesbare Dateien muss für Backups und zum Datenaustausch möglich sein. 

\textcolor{YellowGreen}{Was soll denn ein selektiver Import sein?}

\textcolor{NavyBlue}{Zugegeben ist in diesem Satz die Formulierung unter Umständen unglücklich gewählt. Es soll möglich sein Datensätze z.B. aus bereits bestehenden Backups wieder in die Anwendung einzuspielen. Dabei sollen die gesamten betroffenen Datensätze ausgetauscht werden, aber nicht zwingend die gesamte Datenbasis.}

\textcolor{YellowGreen}{Was verstehen Sie unter selektiven Export?}

\textcolor{NavyBlue}{Es soll möglich sein Datensätze auszuwählen. Z.B. einzelne Mitglieder oder eine Gruppe an Mitgliedern, die gewisse Kriterien erfüllen.}

\textcolor{YellowGreen}{Was wollen Sie selektieren können, was nicht?}

\textcolor{NavyBlue}{An sich sollen alle Datensätze selektiert werden können, die Mitgliederdaten, Termin- und Buchungsdaten, sowie Rechnungsdaten enthalten. Kurz gesagt alle Daten, die auch für eine sinnvolle Erstellung von Backups benötigt werden. }

\textcolor{Red}{Die Frage würd ich rausnehmen da die Formulierung mal wieder 0 Sinn macht und das von den Fragen die ich hinzugefügt hab auch erläutert wird.}
\textcolor{YellowGreen}{Sollen die Backups wirklich händisch selektiert werden oder automatisiert laufen?}

\textcolor{NavyBlue}{Das selektieren soll auf alle Fälle auch händisch möglich sein, da wir den Datenexport auch in anderen Fällen nutzen. Die Erstellung von Sicherheitskopien soll hauptsächlich automatisch laufen, doch auch manuell durchgeführt werden, sollte außerplanmäßig eine Erstellung nötig sein.}

\textcolor{YellowGreen}{Wo und wie werden die Backups gespeichert?}

\textcolor{NavyBlue}{Wir haben einige externe Festplatten, die wir für die Überspielung der Backups an den Server anschließen. Die Aufbewahrung erfolgt im Firmentresor im Büro der Geschäftsführung. }

\textcolor{YellowGreen}{Wie oft werden Backups durchgeführt?}

\textcolor{NavyBlue}{Ein mal wöchentlich soll ein neues Backup erstellt werden.}

\textcolor{YellowGreen}{Sollen die Backups automatisch oder manuell durchgeführt werden?}

\textcolor{NavyBlue}{Die Backups sollen automatisch von der Anwendung ausgeführt werden. Der Zeitpunkt des Backups soll von einem Administrator einstellbar sein. Außerdem soll ein Administrator auch jederzeit manuell ein Backup erstellen können.}

\textcolor{YellowGreen}{Sollen im Backup immer nur die Änderungen zum vorherigen Stand gespeichert werden oder soll immer ein vollständiges Backup erstellt werden?}

\textcolor{NavyBlue}{Es sollen immer vollständige Backups erstellt werden.}

\textcolor{YellowGreen}{Wie lange soll ein Backup aufbewart werden?}

\textcolor{NavyBlue}{Die Backups sollen jeweils für ein Jahr gespeichert werden.}

%\textcolor{YellowGreen}{Sollen die Backups verschlüsselt werden?}

%\textcolor{NavyBlue}{Unsere Systeme sind bereits so gut abgesichert, dass kein Zugriff von außerhalb auf den Speicherort der Backups möglich ist. Deshalb müssen die Backups nicht verschlüsselt werden.}

\textcolor{YellowGreen}{Gibt es redundante Notfallsysteme?}

\textcolor{NavyBlue}{Nein, es gibt keine zusätzlichen Systeme zum einzigen Server im Keller.}

\textcolor{YellowGreen}{In welche Form wollen Sie die Daten für den Datenaustausch exportieren?}

\textcolor{NavyBlue}{Wir kennen uns mit den technischen Aspekten nicht so tief auf, da dies in unserem alten System nicht in diesem Umfang möglich war. Es sollte aber ein Format sein, dass auch für die Mitarbeiter lesbar sein sollte.}

\textcolor{YellowGreen}{Zu welchen Zwecken betreiben Sie Datenaustausch?}

\textcolor{NavyBlue}{Neben dem Export zur Erzeugung von Sicherungskopien haben wir noch eine andere wichtige Verwendung. Wir wollen verschiedene Analyseanwendungen verwenden, die uns die aktuelle Geschäftssituation darstellen und analysieren kann. Dazu brauchen wir die Möglichkeit die Daten in einer von beiden Seiten unterstützten Form von einem Programm ins andere zu transferieren.}

\textcolor{YellowGreen}{Mit welchen Sicherheitsrichtlinien sollen die Backup-/Datenaustauschdateien versehen werden?}

\textcolor{NavyBlue}{Da die Backups und der Datenaustausch keinerlei Kundenkontakt aufweißt, sind bis jetzt keinerlei Sicherheitsvorkehrungen ergriffen worden. Es wäre jedoch gut, wenn nur die Mitarbeiter in der Lage sind, diese Datein und Daten zu betrachten.}

Eine intuitive, leicht bedienbare Benutzeroberfläche setzen wir als selbstverständlich voraus. 
Es sollen keine besonderen Computerkenntnisse zur Bedienung der Software erforderlich sein.  

\textcolor{YellowGreen}{Was ist Ihre Definition von 'keine besonderen Computerkenntnisse'?}

\textcolor{NavyBlue}{Mit einem Firmendurchschnittsalter von 47 Jahren beschäftigen wir statistisch betrachtet überdurchschnittlich viele digitale Quereinsteiger, die noch in einer Zeit ohne Handy und Computer aufgewachsen sind und daher nicht intuitiv mit der Technologie umgehen können. }

\textcolor{YellowGreen}{Was verstehen Sie unter leicht bedienbar, wollen Sie Icons oder Labeltexte?}

\textcolor{NavyBlue}{Die Bedeutung von Icons hat in unserem alten System einerseits zu Verwirrung gesorgt, aber mittlerweile sind unsere Mitarbeiter daran gewöhnt. Eine ausgewogene und verständliche Kombination aus beiden Teilen wäre hier gefordert. Es wäre gut, dass sie gleiche oder ähnliche Symbole wie im SAP System verwenden.}

\textcolor{YellowGreen}{Muss die Benutzeroberfläche auch barrierefrei sein?}

\textcolor{NavyBlue}{Aktuell beschäftigen wir keine Menschen mit Sehbeeinträchtigung, doch soll dieser Fall nicht ausgeschlossen sein, ist aber aktuell nicht zu fokussieren.}

\textcolor{YellowGreen}{Muss es besondere Farb-Modi für z.B. Farbenblindheit, etc. geben?}

\textcolor{NavyBlue}{Auf Anbitte unserer IT-Abteilung wäre es gut, wenn die Anwendung auch so ein "Dark Theme" besitzt}

\subsection{Anwendungsbereiche}

Die Software soll ausschließlich für die Verwaltung von Fahrzeugen, Kunden, Ausrüstung, Fahrzeugstandorte und Angestellten und den damit direkt verbundenen Elementen verwendet werden. 
Sie soll im Alltag auf Desktop-Rechnern und Laptops eingesetzt werden.  

\textcolor{YellowGreen}{Welches Betriebssystem wird erwünscht oder gefordert?}

\textcolor{NavyBlue}{Da Windows XP doch bereits seit Jahren veraltet ist, wäre es gut, Windows 10 zu verwenden.}

\textcolor{YellowGreen}{Es ist wirklich keine mobile Nutzung erwünscht, um spontan von unterwegs Autos buchen zu können?}

\textcolor{NavyBlue}{Ob eine mobile Anwendung benötigt wird, wird von unserer Marktforschungs-Abteilung momentan evaluiert und kann für dieses Projekt vernachlässigt werden.}

\textcolor{YellowGreen}{Gibt es echtzeikritische Aspekte in der Anwendung, die besonders hervorgehoben werden müssen?}

\textcolor{NavyBlue}{Das System sollte schon an einem schnellen Ablauf orientiert sein. Es ist nötig, dass die Terminbuchungen und Rechnungsbelege nahezu sofort umgesetzt werden, damit es keine Problem bei der Terminverarbeitung geben kann. Zusätzlich soll das Buchen von einem Fahrzeug mithilfe eines Reservations-Zeitfensters nur von einer Person auf einmal zugelassen werden (Ähnlich wie zum Beispiel bei Ticketmaster oder Eventim). Das Fenster soll 15 Minuten lang sein, dann soll die Reservierung freigegeben werden, falls sie nicht beendet wurde. }

\textcolor{YellowGreen}{Soll das Programm im Browser aufgerufen werden oder eine Anwendung sein?}

\textcolor{NavyBlue}{Bevor wir uns neben der Desktop-Anwendung für eine umfangreiche Digitalisierung mit Webanwendung und mobiler Handy-App entscheiden, wollen wir zuerst diese Anwendung im Unternehmensalltag integrieren. Im weiteren Verlauf ist jedoch eine Webanwendung gewünscht. Es wäre also erforderlich, dass einige Vorschläge und Konzepte bzgl. einer Einbindung einer Webanwendung bereits erstellt werden.}

\subsection{Zielgruppen, Benutzerrollen und Verantworklichkeiten}

Es soll verschiedene Benutzerrollen geben: 

\begin{itemize}
    \item Organisatorinnen und Organisatoren pflegen die jeweiligen Buchungsdaten und Fahrzeuge. 
    
    \textcolor{YellowGreen}{Wer ist Organisator?}

    \textcolor{NavyBlue}{Als Organisator*in gelten alle Mitarbeiter der zuständigen Abteilung, also nicht die Personalmitarbeiter. Dazu gehören die Mitarbeiter*innen, die in den Filialen mit Kundenkontakt arbeiten, da diese die Buchungsdaten eingeben müssen.}

    \textcolor{YellowGreen}{Kann jeder Organisator alles?}

    \textcolor{NavyBlue}{Organisatoren haben vollen Zugriff auf Buchungs- und Fahrzeugdaten, Lesezugriff auf Rechnungsdaten und keinen Zugriff auf etwaige Personaldaten.}

    \textcolor{YellowGreen}{Woraus bestehen die Buchungsdaten?}

    \textcolor{NavyBlue}{
        \begin{itemize}
            \item Referenz auf das jeweilige Kundenkonto
            \item Zeitraum der Buchung
            \item das gebuchte Fahrzeug
            \item der Standort
            \item Verweis auf die zum Kundenkonto gehörenden Rechnungsdaten 
        \end{itemize}
    }

    \textcolor{YellowGreen}{Wie sollen die Daten gespeichert werden?}

    \textcolor{NavyBlue}{Die Daten sollen als zusammengehörender Datensatz wie auch alle anderen Daten gespeichert werden. (siehe oben)}

    \item Personalmitarbeiter pflegen Mitarbeiterdaten im System  
    
    \textcolor{YellowGreen}{Pflegen die Personalmitarbeiter auch ihre eigenen Daten?}

    \textcolor{NavyBlue}{Da wir vollstes Vertrauen in unsere Mitarbeiter haben, ist die Bearbeitung der eigenen Mitarbeiterdaten möglich.}

    \textcolor{YellowGreen}{Gibt es Einschränkungen, ob bestimmte Daten nicht bearbeitet werden dürfen?}

    \textcolor{NavyBlue}{Wichtig ist, dass vergebene IDs, Rechnungsnummern, Buchungsnummern, etc. nach Erstellung und Eintragung nicht mehr geändert werden dürfen und eindeutig sind. }

    \textcolor{YellowGreen}{Dürfen Personalmitarbeiter auch andere Daten einsehen, z.B. Buchungsdaten?}

    \textcolor{NavyBlue}{Wir arbeiten nach Vorbild aus anderen Firmen nach dem 'Need-To-Know'-Prinzip. Somit ist es nicht nötig, dass die Personalmitarbeiter auf andere Datensätze als die Personalstammdaten editierend zugreifen können.}

    \item Eine hauptverantwortliche Person (Administrator) hat Vollzugriff auf sämtliche Daten, vor allem für deren Import und Export sowie deren Backup. 
    
    \textcolor{YellowGreen}{Vergibt der Admin die Rechte für die Anderen? Wenn nein: Wer verteilt sonst die Rechte?}

    \textcolor{NavyBlue}{Die Aufgabe der Administratorrolle ist die umfassende Verwaltung der Software. Dazu gehört die Rollenzuweisung und somit ist ein Administrator auch für die Verteilung der Zugriffsrechte zuständig.}

    \textcolor{YellowGreen}{Heißt Vollzugriff auch unbeschränktes ändern?}

    \textcolor{NavyBlue}{Vollzugriff bedeutet unbeschränkter Umgang mit allen Daten. Davon ausgeschlossen sind jedoch die bereits erwähnten Daten zur Identifizierung verschiedener Objekte wie Buchungen oder Rechnungen.}

    \textcolor{YellowGreen}{Welche Sicherheitsvorkehrungen sollen getroffen werden, sodass kein Datenmissbrauch möglich ist?}

    \textcolor{NavyBlue}{Es muss mindestens zwei Administratoren geben und bei Zuweisung oder Änderung von Rechten und Rollen muss ein anderer Administrator dieser Änderung zustimmen, bevor sie umgesetzt wird.}

    \item Es gibt keine Gruppen oder Abteilungen, die verwaltet werden müssen.
    
    \textcolor{YellowGreen}{Wenn es keine Unterschiedlichen Abteilungen gibt, woran wird die Rollenvergabe festgemacht?}

    \textcolor{NavyBlue}{Wir möchten, dass die Rollenvergabe händisch erfolgt. Der Administrator soll einzelnen Personen oder einer Mehrfachauswahl die Rollen anpassen können. Wichtig dabei ist, dass es zu Beginn die benötigten Systembenutzer gibt, mit denen man die initiale Rollenvergabe durchführen kann.}

\end{itemize}

\textcolor{YellowGreen}{Welche 'Rolle' soll Kunden zugewiesen werden?}

\textcolor{NavyBlue}{Für die Verwendung der Desktop-Anwendung ist es nicht nötig, dass es einen extra Kundenzugriff gibt, doch in Berücksichtigung einer möglichen Webanwendung, wo auch die Mitglieder individuell ihre Buchungen tätigen können, sollten die Mitglieder dann einfach die Rolle 'Kunde' erhalten.}


\subsection{Zusammenspiel mit anderen Systemen}

Die Daten über die Angestellten (Gehälter bzw. Löhne, Steuern, Kranken- und Rentenversicherung usw.) werden separat durch ein vorhandenes Personalbuchhaltungsprogramm verwaltet und müssen hier nicht berücksichtigt werden. Die finanztechnischen Daten werden über unser vorhandenes Finanzsystem erfasst und müssen hier ebenfalls nicht berücksichtigt werden. 

\textcolor{YellowGreen}{Welche Personaldaten werden dann verwaltet? (siehe 5.2.3 PErsonalmitarbeiter)}

\textcolor{YellowGreen}{Welche Kunden-/Buchungsdaten werden verwaltet?}

Die Software soll aus zwei Teilen bestehen:  

\begin{itemize}
    \item Für die Mitarbeiter im Büro soll eine Desktop-Anwendung erstellt werden, mit denen die Datenbestände verwaltet werden können. Es sollen auch Buchungen erstellt werden können für Kunden, die keine Online-Buchungen machen wollen und persönlich in der Carsharing-Filiale erscheinen. 
    
    \textcolor{YellowGreen}{Gibt es Beschränkungen bei Schreibzugriffen?}

    \textcolor{YellowGreen}{Wie soll bei Kunden abgerechnet werden?}

    \textcolor{YellowGreen}{Gibt es Online-Banking, Kartenzahlung, etc.?}

    \textcolor{YellowGreen}{Soll für Kunden, die in der Filiale buchen, auch eine Buchung auf Rechnung möglich sein?}

    \textcolor{YellowGreen}{Wie soll zwischen verschiedenen Rollen unterschieden werden? Über Login, RSA Token, Zweifaktor-Authentifizierung, etc.?}

    \item Eine neue Web-Seite soll unseren Online-Kunden ermöglichen, nach einer Authentifizierung alle Standorte anzeigen zu lassen sowie natürlich die dort befindlichen Fahrzeuge, welche für einen anzugebenden Zeitbereich online gebucht werden können. 
    
    \textcolor{YellowGreen}{Soll die Buchung (visuell) identisch zu der Desktop-App stattfinden oder für Webanwendungen angepasst sein?}

    \textcolor{YellowGreen}{Soll vorher der Zeitbereich angegeben und nur verfügbare Fahrzeuge angezeigt werden oder soll erst ein Fahrzeug und dann der verfügbare Zeitraum gezeigt werden?}
\end{itemize}

Die Web-Seite soll mit dem ersten Teilauftrag noch nicht programmiert werden, allerdings benötigen wir ein klares Konzept, wie diese Web-Seite realisiert werden soll (Schnittstellen usw.).  

\textcolor{YellowGreen}{Wie detailliert sollen die Mock-Ups erstellt werden?}

Möglichst alle Daten sollen vom alten in das neue System übertragen werden. 
\newpage

\subsection{Produktfunktionen}

\begin{tabular}{l | p{13cm}}
    \hline
    /LF10/ & Der jeweilige Benutzer muss die Möglichkeit haben, über eine grafische Benutzeroberfläche alle für ihn relevanten Daten einfach und übersichtlich zu verwalten. 

    \textcolor{YellowGreen}{Wie werden Kunden von Mitarbeitern unterschieden? Unterscheidung durch Login?}

    \textcolor{NavyBlue}{}

    \textcolor{YellowGreen}{Wie sind die Benutzergruppen definiert?}

    \textcolor{NavyBlue}{}

    \textcolor{YellowGreen}{Was sind relevante Daten?}

    \textcolor{NavyBlue}{}

    \textcolor{YellowGreen}{Was heißt einfach und übersichtlich?}

    \textcolor{NavyBlue}{}

    \textcolor{YellowGreen}{Sollen firmenspezifische Symbole, Farbmuster, etc. integriert werden oder ist dies eher nebensächlich?}

    \textcolor{NavyBlue}{}

    Es sollen zahlreiche Konfigurationsdaten gespeichert und beim nächsten Start des Programms verwendet werden (z.B. aktuelle Größe und Position des Fensters). Daneben sollen einige Elemente vor dem Start konfigurierbar sein (z.B. Überschriften, Schriftarten und -größen usw.). 
    
    \textcolor{YellowGreen}{Was ist mit Unterschriften gemeint?}

    \textcolor{NavyBlue}{}

    \textcolor{YellowGreen}{Kann die Größe von Überschrift und restlichem Text insgesamt frei gewält werden oder nur im Zusammenhang? Groß, mittel, klein... (Bei freier Wahl könnte eine kleinere Schriftgröße für die Überschrift gewählt werden, wie für den Text).}

    \textcolor{NavyBlue}{}

    \textcolor{YellowGreen}{Welche Konfigurationsdaten genau? (Farbschema, Shortcuts, etc.?)}

    \textcolor{NavyBlue}{}

    \textcolor{YellowGreen}{Constraints -> Sollen die Werte dazu frei auswählbar oder in einem gewissen Rahmen vorgegeben sein?}

    \textcolor{NavyBlue}{}

    \textcolor{YellowGreen}{Sollen die Konfigurationsdaten anwendungsspezifisch oder benutzerspezifisch sein?} 
    
    \textcolor{NavyBlue}{}
    \\
    \hline
    /LF20/ & Verwaltet werden sollen Mitarbeiter, Fahrzeuge, Standorte, Kunden, Buchungen, Rechnungen, Änderungen, Stornierungen, Mahnungen usw.

    \textcolor{YellowGreen}{Welche Benutzergruppen soll es geben?}

    \textcolor{NavyBlue}{}

    \textcolor{YellowGreen}{Welche Gruppe hat Zugriff auf welche Daten?}

    \textcolor{NavyBlue}{}

    \textcolor{YellowGreen}{Was ist usw.?}

    \textcolor{NavyBlue}{}

    \textcolor{YellowGreen}{Welche Daten soll für Mitarbeiter, Fahrzeuge, Standorte, Kunden, Buchunger, Rechnungen, Änderungen, Stornierungen und Mahnungen erhoben werden?}

    \textcolor{NavyBlue}{}

    \textcolor{YellowGreen}{Worin unterscheiden sich Buchung und Rechnung?}

    \textcolor{NavyBlue}{}

    \textcolor{YellowGreen}{Wird eine Rechnung aus den Buchungsdaten generiert?}

    \textcolor{NavyBlue}{}

    \textcolor{YellowGreen}{Gibt es eine Mindestlänge für einen gebuchten Termin?}

    \textcolor{NavyBlue}{}

    \textcolor{YellowGreen}{Gibt es eine Maximallänge für einen gebuchten Termin?}

    \textcolor{NavyBlue}{}

    \textcolor{YellowGreen}{Gibt es Rabatte für längere Buchungen? Ab wann gilt eine Buchung als länger?}

    \textcolor{NavyBlue}{}

    Es muss möglich sein, jederzeit erkennen zu können, welche angestellte Person   einen Datensatz angelegt, geändert oder gelöscht hat. 
    
    \textcolor{YellowGreen}{Müssen alle Änderungen sichtbar sein oder nur die Letzte?}

    \textcolor{NavyBlue}{}
    \\
    \hline
    /LF30/ & Buchungen haben eine Start- und einen Endtermin, Terminüberschneidungen müssen vermieden werden, um die Verfügbarkeit sicherzustellen. 
    
    \textcolor{YellowGreen}{Was wenn der vorherige Kunde sich verspätet? Wird ein Puffer eingeplant?}

    \textcolor{NavyBlue}{}

    \textcolor{YellowGreen}{Gibt es Wartungsfenster zwischen Buchungen für Reinigungen, etc.?}

    \textcolor{NavyBlue}{}

    \textcolor{YellowGreen}{Was wenn Mängel während der Fahrt auftreten?}

    \textcolor{NavyBlue}{}

    \textcolor{YellowGreen}{Sollen nicht verfügbare Fahrzeuge als solche gekennzeichnet werden, z.B. durch eine graue Darstellung?}

    \textcolor{NavyBlue}{}

    \textcolor{YellowGreen}{Wie soll die Befüllung der Tanks gewährleistet werden?}

    \textcolor{NavyBlue}{}

    \textcolor{YellowGreen}{Kann es passieren, dass der Standort des Fahrzeuges geändert werden muss?}

    \textcolor{NavyBlue}{}
    \\
    \hline
    /LF40/ & Unsere Kunden haben neben ihren Kontaktdaten auch Vertragsunterlagen für die Teilnahme am Carsharing, die die Höhe des Eigenanteils für einen Schadensfall der Versicherung sowie die Höhe der Teilnahme-Kaution enthält. Diese Vertragsunterlagen werden von uns eingescannt und sollen als Dokument mit den Kundendaten gespeichert werden. Daneben wird jedem Kunden eine Karte zum Öffnen und Schließen der Fahrzeuge ausgehändigt. 

    \textcolor{YellowGreen}{Wie werden Schließberechtigungen verwaltet?}

    \textcolor{NavyBlue}{}

    \textcolor{YellowGreen}{Ist die Karte spezifisch für das Fahrzeug?}

    \textcolor{NavyBlue}{}

    \textcolor{YellowGreen}{Welche Daten sind auf der Kundenkarte gespeichert?}

    \textcolor{NavyBlue}{}

    \textcolor{YellowGreen}{Wo und in welcher Form sind die Kundendaten gespeichert? Magnetband, Chip oder QU-Code?}

    \textcolor{NavyBlue}{}

    \textcolor{YellowGreen}{Sollen die Daten auf der Karte bearbeitet werden können?}

    \textcolor{NavyBlue}{}

    \textcolor{YellowGreen}{Sollen die Daten beim Scan manuell oder automatisch konvertiert werden?}

    \textcolor{NavyBlue}{}

    \textcolor{YellowGreen}{Wie lange müssen die Vertragsunterlagen und Buchungsbelege aufbewahrt werden?}

    \textcolor{NavyBlue}{}

    \textcolor{YellowGreen}{Welche Art der Langzeitspeicherung gibt es? Magnetband, Festplatten, etc.?}

    \textcolor{NavyBlue}{}

    Ein kleiner Prozessor im Fahrzeug sendet nach Fahrtende (Terminende) die exakte Start- und Ende-Zeit sowie die gefahrenen Kilometer an einen Server. Diese Daten sollen von dem Server nach Buchungsende geholt und zur Berechnung der Kosten für die Buchung (und somit für die Rechnung) verwendet werden. 
    
    \textcolor{YellowGreen}{Soll dafür http oder ein eigenes Protokoll verwendet werden? Muss Verschlüsselung beachtet werden?}

    \textcolor{NavyBlue}{}

    \textcolor{YellowGreen}{Was passiert, wenn das Fahrzeug zu Terminende noch fährt?}

    \textcolor{NavyBlue}{}
    \\
    \hline
\end{tabular}

\begin{tabular}[ht] {l | p{13cm}}
    \hline
    /LF50/ & Die Fahrzeuge selbst gehören unterschiedlichen Kategorien an: 

    Kleinfahrzeuge, Mittelklassefahrzeuge, gehobene Mittelklasse und Transportfahrzeuge.  

    Allen Kategorien sind eine bestimmte Höhe der Stunden-Mietpauschale und die Kosten pro gefahrenem km zugeordnet. Alle Werte sollen konfigurierbar sein. 

    \textcolor{YellowGreen}{Wie sollen die Daten konfigurierbar bleiben? Sollen Preisänderungen über die Datenbankebene oder mittels GUI erfolgen?}

    \textcolor{NavyBlue}{Die Preisänderung soll auf Datenbankebene mithilfe einer Konfigurationsdatei geschehen.}

    \textcolor{YellowGreen}{Gibt es Events wie Angebotswochen?}

    \textcolor{NavyBlue}{Dies ist in Zukunft auf jeden Fall geplant, jedoch Bedarf dies auch nur einer Änderung der Preise in der Konfigurationsdatei.}

    \textcolor{YellowGreen}{Können Mitglieder Prämien sammeln, die zu Vergünstigungen oder ähnliches führen?}

    \textcolor{NavyBlue}{Unsere Kunden können für eine Fahrt im Wert von mindestens 10€ Payback-Punkte sammeln. Dies ist jedoch unabhängig von der zu implementierenden Anwendung.}

    \textcolor{YellowGreen}{Ist diese Stunden-Mietpauschale nur für die Fahrzeugart treffend oder unterscheidet sie sich auch innerhalb dieser Kategorien je nach Auto?}

    \textcolor{NavyBlue}{Die Pauschale ist für alle Fahrzeuge innerhalb einer Kategorie gleich.}

    Alle Fahrzeuge werden regelmäßig von Fremdfirmen gewartet. Die entsprechenden Dienstleistungen sollen den Fahrzeugen chronologisch zugeordnet werden. 
    
    \textcolor{YellowGreen}{Soll es eine Rechnungsverwaltung geben? Werden die Rechnungsdaten im gleichen System gespeichert oder müssen diese separat verwaltet werden?}

    \textcolor{NavyBlue}{Die Rechnungsdaten werden separat verarbeitet. Es soll lediglich vermerkt werden, welcher Service wann durchgeführt wurde.}

    \textcolor{YellowGreen}{Was ist mit chronologischer Zuordnung gemeint?}

    \textcolor{NavyBlue}{Die Dienstleistungen sollen nach Datum geordnet angezeigt werden. Die aktuellste Dienstleistung soll dabei zuerst angezeigt werden.}
    \\
    \hline
    /LF60/ & Einem Standort können ein oder mehrere Fahrzeuge zugeordnet sein. Ein Fahrzeug ist immer nur einem Standort zugeordnet. 
    
    \textcolor{YellowGreen}{Welche Standorte gibt es?}

    \textcolor{NavyBlue}{Da wir planen großflächig nach Fidji zu expandieren muss das Anlegen neuer Standorte in der Anwendung möglich sein. Die bereits bestehenden Standorte werden dann bei der Inbetriebnahme von unseren Praktikanten angelegt und müssen demzufolge nicht betrachtet werden.}

    \textcolor{Red}{Das is ja nichtmal ein Satz hier. Macht für mich keinen Sinn bzw Zusammenhang fehlt.}
    \textcolor{YellowGreen}{Kennzeichnungen für Standorte und Fahrzeuge?}

    \textcolor{NavyBlue}{}
    
    \textcolor{YellowGreen}{Gibt es ein (Live-) Fahrzeugtracking?}
    
    \textcolor{NavyBlue}{Da dies ein Eingriff in die Privatsphäre unserer Kunden wäre möchten wir auf diese Funktion verzichten.}
    \\
    \hline
    /LF70/ & Nach jeder Fahrt werden sofort die Rechnungen erstellt und dem Kunden per  
    E-Mail zugesandt. 

    \textcolor{YellowGreen}{Könnten bei der Sendung von Rechnungsdaten per Email als Klartext eventuell DSGVO-Probleme auftreten?}

    \textcolor{NavyBlue}{Ja, aber da unser Firmensitz in Dubai liegt ist uns das egal. Wir machen das einfach so wie Telegram.}

    \textcolor{YellowGreen}{Sollten die Rechnungsdaten nicht erst an den Webserver gesendet und dann dort verarbeitet werden?}

    \textcolor{NavyBlue}{Für einen Webserver reicht unser Budget leider nicht aus. Deshalb muss es auch ohne diese Technologie funktionieren.}

    \textcolor{YellowGreen}{Was ist die Zeitraumdefinition für 'sofort'?}

    \textcolor{NavyBlue}{Wenn der Kunde die Fahrt beendet soll automatisert eine Rechnung erstellt werden, welche direkt an den Kunden geschickt wird. Die Rechnung sollte maximal 10 Minuten nach dem Ende der Fahrt beim Kunden ankommen.}

    Die Bezahlung der Rechnungen erfolgt über Bankeinzug, was durch das Finanzbuchhaltungssystem (FBH) erledigt wird und hier nicht betrachtet werden muss. Allerdings muss über die vorhandene Schnittstelle des FBH der Stand der Rechnungsbegleichung abgefragt werden, damit über das neue System erkennbar ist, ob und wann eine Rechnung bezahlt wurde. 
    


    \textcolor{YellowGreen}{Wie regelmäßig soll die Abfrage stattfinden? 1x pro Tag, mehrmals täglich?}

    \textcolor{NavyBlue}{Es soll einen Aktualisierungsknopf für geben, welcher eine Anfrage an die Schnittstelle sendet und die Daten aktualisiert. Zusätzlich sollen alle 30 Minuten die Daten automatisch abgefragt werden.}

    \textcolor{YellowGreen}{Gibt es für die Rechnungen bereits Rechnungsnummern, oder müssen diese erst erstellt werden? Welches System wird für die Nummernvergabe verwendet?}
    
    \textcolor{NavyBlue}{Die Vergabe der Rechnungsnummern geschieht automatisch durch das FBH und muss daher nicht betrachtet werden.}
    \\
    \hline
    /LF80/ & Buchungen können bis 10 Stunden vor Antritt der Fahrt storniert werden. 
    
    \textcolor{YellowGreen}{Wie soll die Meldung des Systems aussehen, wenn die 10h-Marke unterschritten wird?}

    \textcolor{NavyBlue}{Es soll ein Popup-Fenster geöffnet werden, welches den Nutzer darüber informiert.}

    \textcolor{YellowGreen}{Soll die Funktion "Stornieren" nicht mehr angezeigt werden, falls die 10h-Marke unterschritten wird?}

    \textcolor{NavyBlue}{Der Knopf zum stornieren soll weiterhin angezeigt werden, jedoch soll er entweder deaktiviert sein oder beim Betätigen soll eine Fehlermeldung per Popup angezeigt werden.}

    \textcolor{YellowGreen}{Impliziert diese Einschränkung auch, dass Buchungen unter der 10h-Grenze nicht mehr möglich sind oder einfach nur nicht storniert werden können?}

    \textcolor{NavyBlue}{Ja. Da die Buchung von unseren Mitarbeitern freigeschalten werden muss ist auch hierfür eine Grenze von 10h einzubauen.}
    \\
    \hline
    /LF90/ & Zur einfacheren Eingabe der Daten soll es Auswahllisten für deren Eigenschaften geben, wo immer es möglich ist. Die Auswahllisten sollen auf einfache Weise erweiterbar sein. 
    %------------------------------------------------------
    \textcolor{YellowGreen}{Welche Eingabefelder gibt es?}

    \textcolor{NavyBlue}{}

    \textcolor{YellowGreen}{Welche dieser Eingaben sollen mithilfe von Auswahllisten umgesetzt werden?}

    \textcolor{NavyBlue}{}
    %-----------------------------------------------------
    \textcolor{YellowGreen}{Wie soll die Listenerweiterung vorgenommen werden? Reicht ein einfacher "Edit"-Knopf?}

    \textcolor{NavyBlue}{Es soll eine Bearbeitungsansicht für das komplette Formular geben, welche über einen "Edit"-Knopf erreichbar sein soll. In dieser Ansicht soll für jede Auswahlliste ein Textinput mit einem Knopf zum Hinzufügen bereitgestellt werden.}

    \textcolor{YellowGreen}{Wer hat die Berechtigungen, diese Liste zu erweitern?}

    \textcolor{NavyBlue}{Die Berechtigungen für die Erweiterung der Auswahllisten sind auf die Administratoren beschränkt.}

    \textcolor{YellowGreen}{Kann die Liste im vollen Umfang verändert werden oder soll es noch Rahmenbedingungen geben?}

    \textcolor{NavyBlue}{}

    \textcolor{YellowGreen}{Ist eine Mehrfachauswahl oder Mehrfachbearbeitung möglich?}

    \textcolor{NavyBlue}{Nein. Auf }
    \\
    \hline
    /LF100/ & Sämtlichen Elementen sollen mehrere Bilder mit Titel zugeordnet werden können, die zentral auf einem Verzeichnis liegen sollen 
    
    \textcolor{YellowGreen}{Welche Vorgaben gibt es für die Bilder? Bildformat, Dateiformat, etc.}

    \textcolor{NavyBlue}{Die Bilder sollen im JPEG- oder PNG-Format vorliegen.}

    \textcolor{Red}{Berechtigungen wofür?}
    \textcolor{YellowGreen}{Wer hat die dazu benötigten Berechtigungen?}

    \textcolor{NavyBlue}{}

    \textcolor{Red}{da steht doch das es zentral in einem Verzeichnis gespeichert werden soll also ich versteh ni was die frage soll}
    \textcolor{YellowGreen}{Gibt es ein zentrales Verzeichnis?}

    \textcolor{NavyBlue}{}

    \textcolor{YellowGreen}{Ist ein einheitliches Design für die Seiten vorgegeben?}

    \textcolor{NavyBlue}{Das Design wird nicht vorgegeben aber es sollte auf allen Seiten einheitlich sein.}

    \textcolor{YellowGreen}{Gibt es eine maximale Anzahl an Bildern pro Element?}

    \textcolor{NavyBlue}{Die Anzahl der Bilder pro Element soll auf 20 beschränkt sein.}

    \textcolor{YellowGreen}{Gibt es eine maximale Dateigröße für die Bilder, um Ladezeiten und Speicher einzugrenzen?}

    \textcolor{NavyBlue}{Die Dateigröße ist auf 10 MB pro Bild beschränkt.}
    \\
    \hline
\end{tabular}

\subsection{Produktdaten}

\begin{center}
    \begin{tabular}[ht] {l | p{13cm}}
        \hline
        /LD10/ & Die Daten sollen zunächst in einer zentralen Datenbasis (lesbare Dateien) abgespeichert und später in eine Datenbank überführt werden. 
        
        \textcolor{YellowGreen}{Wo ist diese Datenbasis, wie sieht diese Datenbasis aus?}
        
        \textcolor{NavyBlue}{Die Datenbasis besteht aus CSV-Dateien, welche vorerst lokal bereitgestellt werden.}


        \textcolor{YellowGreen}{Welches Dateiformat wird benötigt?}

        \textcolor{NavyBlue}{Als Dateiformat wird CSV benötigt.}

        \textcolor{Red}{->}
        \textcolor{YellowGreen}{Was bedeutet lesbar? Für Menschen, für Computer, welche Sprache (Englisch oder Deutsch)?}

        \textcolor{NavyBlue}{Lesbar bedeutet in diesem Zusammenhang, dass die Leserechte für diese Dateien nicht eingeschränkt beziehungsweise für die Anwendung freigegeben sind.}

        \textcolor{Red}{->}
        \textcolor{YellowGreen}{Wer soll die Dateien lesen können?}
        
        \textcolor{NavyBlue}{Die in den Dateien enthaltenen Daten sollen von der Anwendung eingelesen und anschaulich dargestellt werden.}


        \textcolor{YellowGreen}{Um welche Daten handelt es sich?}

        \textcolor{NavyBlue}{Bei den Daten handelt es sich um die Daten der Mitglieder sowie deren Buchungen.}


        \textcolor{YellowGreen}{In was für eine Datenbank sollen die Daten überführt werden?}

        \textcolor{NavyBlue}{Die Daten sollen in Zukunft in einer SQL-Datenbank gespeichert werden. Jedoch ist dies noch nicht in naher Zukunft geplant und kann deshalb vorerst vernachlässigt werden.}

        \textcolor{YellowGreen}{Wann ist später, sollen Vorbereitungen für die Überführung getroffen werden?}

        \textcolor{NavyBlue}{Die Überführung soll kurz vor der Inbetriebnahme der Anwendung geschehen, jedoch sehen wir keine Notwendigkeit dafür Vorbereitungen zu treffen, da unsere IT-Abteilung kompetent genug ist dies eigenständig umzusetzen.}

        \textcolor{YellowGreen}{Wie sieht die Backupstrategie für Datenbasis und Datenbank aus?}

        \textcolor{NavyBlue}{Um die Datenbank kümmert sich wie bereits erwähnt unsere IT-Abteilung. Da die Datenbasis sowieso abgelöst werden soll, müssen hier keine Backups erstellt werden.}

        \textcolor{Red}{Bitte was? Wo steht denn separat? Das eine löst das andere doch ab oder bin ich jz lost?}
        \textcolor{YellowGreen}{Warum "Datenbasis" und Datenbank separat? Welche Vorteile soll das haben?}

        \textcolor{YellowGreen}{Gibt es einen Multiuser-Kontext für die "Datenbasis"? Wie sollen gleichzeitige Zugriffe geregelt sein?}
        \textcolor{NavyBlue}{Nein. Da die Datenbasis nur lokal implementiert wird, muss vorerst kein Multiuser-Kontext beachtet werden.}
        \\
        \hline
        
    \end{tabular}
\end{center}

\subsection{Produktleistungen}

\begin{center}
    \begin{tabular}[ht] {l | p{13cm}}
        \hline
        /LL10/ & Die Anzahl der zu verwaltenden Elemente wird auf ca. 100.000 geschätzt. 
        
        \textcolor{YellowGreen}{Kann mit der Elementzahl in den lesbaren Dateien der zentralen Datenbasis sinnvoll umgegangen werden?}

        \textcolor{NavyBlue}{Die zentrale Datenbasis ist für eine sehr große Anzahl von Elementen ausgelegt, damit auch im Falle eines großen Wachstums der Datenmenge weiterhin ein leichter Umgang mit den Daten möglich ist.}

        \textcolor{YellowGreen}{Ist das eine Schätzung für den "aktuellen" Satz? Wird das zukünftige Wachstum beachtet? Werden es signifikant mehr?}
        
        \textcolor{NavyBlue}{Ja. Die 100.000 Elemente sind nur eine Schätzung. Bei der letzten Erhebung, welche vor einem Monat stattgefunden hat, waren es 96.690 Elemente. Unser Unternehmen rechnet mit einem massiven Wachstum, sobald die neue Anwendung in Betrieb genommen wurde.}

        %\textcolor{YellowGreen}{Könnte das ständige Synchronisieren von einer großen Anzahl an Elementen nicht fehleranfällig und zeitaufwändig sein?}

        %\textcolor{NavyBlue}{}

        \\
        \hline
        /LL20/ & Um bei HW- und SW-Anschaffungen und -neuerungen flexibel zu bleiben, ist auf Plattformunabhängigkeit besonders zu achten. 
        
        \textcolor{YellowGreen}{Gibt es Vorzüge wie die Plattformunabhängigkeit angegangen werden soll?}

        \textcolor{NavyBlue}{In Zusammenhang mit der Plattformunabhängigkeit bevorzugen wir eine Java-Applikation, da Java auf allen Firmensystemen bereits installiert ist.}

        \textcolor{YellowGreen}{Soll die Applikation auch mobil genutzt werden? Wird Kompatibilität für Mobilgeräte benötigt?}

        \textcolor{NavyBlue}{Nein. Unsere Firma stellt momentan keine mobilen Endgeräte für die Mitarbeiter zur Verfügung, sondern nur Desktop-Rechner. Daher ist Kompatibilität mit mobilen Endgeräten nicht gefordert.}

        \textcolor{YellowGreen}{Ist eine Cloudlösung/ oder -speicherung denkbar?}

        \textcolor{NavyBlue}{Nein. Unser altes Personal ist im Umgang mit Cloudlösungen nicht geschult. Das Internet ist für uns alle Neuland.}

        \\
        \hline
    \end{tabular}
\end{center}

\subsection{Qualitätsanforderungen}

\begin{center}
    \begin{tabular} {l | c | c | c | c}
        \hline
        Produktqualität & sehr gut & gut & normal & nicht relevant \\
        \hline
        Funktionalität & X & & & \\
        \hline
        Zuverlässigkeit & & X & & \\
        \hline
        Effizienz & & X & & \\
        \hline
        Benutzbarkeit (auch Gestaltung) & X & & & \\
        \hline
        Wartbarkeit & & & X & \\
        \hline
        Übertragbarkeit (Portabilität) & & & X & \\
        \hline
    \end{tabular}
\end{center}

\textcolor{YellowGreen}{Gibt es für die Zuverlässigkeiten bestimmten Konventionen, Standards oder Vorschriften, die eingehalten werden müssen? SLA?}

\textcolor{NavyBlue}{Nein. Wir sind zufrieden, wenn die Anwendung ohne Störungen verwebdet werden kann.}

\textcolor{YellowGreen}{Was bedeutet gute Funktionalität, was soll dabei beachtet werden? Stabilität, unkomplizierte Bedienbarkeit, besonders viele Funktionen oder nur die minimal benötigten Funktionen?}

\textcolor{NavyBlue}{Als funktional betrachten wir eine Anwendung, welche unkompliziert zu Bedienen ist und mit alle Anwendungsszenarien umgehen kann.}


\textcolor{YellowGreen}{Was mach die Anwendung zuverlässig? Stabilität oder Fehlertoleranz?}

\textcolor{NavyBlue}{Die Anwendung soll einfach nicht abstürzen. Dies bezieht sich natürlich auch auf den Umgang mit fehlerhaften Eingaben.}


\textcolor{YellowGreen}{Was macht die Anwendung effizient? Geschwindigkeit oder sparsamer Ressourcenumgang?}

\textcolor{NavyBlue}{Die Effizienz bezieht sich auf sparsamen Ressourcenumgang. In diesem Zusammenhang soll das speichern unnötiger Daten vermieden werden.}


\textcolor{YellowGreen}{Benutzbarkeit soll sehr gut sein, welche Anforderungen stehen dahinter? Durch was wird das Programm leicht bedienbar? Mit welchen Hilfsmitteln kann die Bedienbarkeit erleichter werden (Shortcuts, Gesten)?'}

\textcolor{NavyBlue}{Alle Funktionen des Programms sollen mit maximal 3 Klicks erreichbar sein und es soll auf Transaktionen verzichtet werden. Shortcuts werden nicht benötigt.}


\textcolor{YellowGreen}{Gestaltung soll sehr gut sein? Ist das nötig, wenn in den Büros eigentlich nur die Mitarbeiter arbeiten und die Kunden gar nicht in den Kontakt mit der Anwendung treten?}

\textcolor{NavyBlue}{Die Bedienung der SAP GUI bereitet unseren Mitarbeitern momentan große Kopfschmerzen. Daher wünschen wir uns eine ansprechendere Benutzeroberfläche mit vereinfachter Bedienung, um die Mitarbeiterzufriedenheit zu erhöhen.}


\textcolor{YellowGreen}{Protabiliät ist normal relevant, aber die Plattformunabhängigkeit ist besonders zu beachten? Wie ist das zu verstehen?}

\textcolor{NavyBlue}{Die Anwendung soll plattformunabhängig auf allen gängigen Betriebssystemen eingesetzt werden können. Außerdem kann es in seltenen Fällen vorkommen, dass die Anwendung auf ein anderes System übertragen werden soll. Allerdings muss kein besonderer Fokus auf die erleichterung dieser Übertragung gelegt werden.}